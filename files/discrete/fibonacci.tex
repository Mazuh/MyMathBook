\documentclass[12pt]{article}

\usepackage{amsmath}
\usepackage{amsfonts}
\usepackage{amssymb}
\usepackage[utf8]{inputenc}

% hides or overrides some titling semantics
\newcommand{\subject}[1]{\title{#1}}
\newcommand{\repository}[1]{\author{#1}}
\newcommand{\topic}[1]{\date{#1}}

\subject{Matemática Discreta}
\repository{(https://github.com/Mazuh/MyNotebook-Math)}
\topic{\bf Demonstração da Forma Fechada de Fibonacci}

\begin{document}

\maketitle

A sequência de Fibonacci é gerada por uma {\bf relação de recorrência},
cujo n-ésimo termo é dado por f(n) tal que:

\begin{align*}
&\boldsymbol{ f(0) = 0 };\\
&\boldsymbol{ f(1) = 1 };
\end{align*}
\begin{align}
\boldsymbol{ f(n) = f(n-1) + f(n-2) }, se\ \boldsymbol{ n > 1 }
\end{align}

Observa-se que essa relação é:\\
- {\bf de segunda ordem}, recorrendo aos dois últimos termos;\\
- {\bf linear}, termos recorrentes não fazem produto e
a maior potência é 1;\\
- {\bf homogênea}, o coeficiente independente de termo
recorrente é 0.

Por causa dessas características, talvez seja possível encontrar 
sistematicamente uma solução para essa recorrência usando a
seguinte igualdade:

$$
f(n) = r^n
$$

Que é equivalente a:

$$
f(n) = r^n \implies f(n-1) = r^{n-1} \implies f(n-2) = r^{n-2}\\
$$

E essa técnica nos dá a seguinte {\bf equação característica}:

$$
r^n = r^{n-1} + r^{n-2}
$$

(que pode ser simplificada dividindo tudo pelo $r$ de menor potência)

$$
\frac{r^{n}}{r^{n-2}} = \frac{r^{n-1}}{r^{n-2}} + \frac{r^{n-2}}{r^{n-2}}
$$

(e usaremos a propriedade de divisão de potênciações)

$$
r^{n - (n-2)} = r^{n-1 - (n-2)} + r^{n-2 - (n-2)}
$$
$$
r^{{\bf n-n}+2} = r^{{\bf n}-1{\bf-n}+2} + r^{\bf n-2-n+2}
$$
$$
r^2 = r^1 + r^0
$$
$$
r^2 = r + 1
$$

Ao trazermos tudo ao membro esquerdo, a equação característica fica:
\begin{align}
{\bf r^2 - r - 1 = 0 }
\end{align}

Dizemos que essa equação (2) é característica da recorrência (1) pois
dependemos de suas raízes $r$ para encontrar uma solução à
recorrência $f$. Dito isso, pela fórmula quadrática:

$$
\Delta = (-1)^2 - 4(1)(-1)
$$
$$
= 1 + 4
$$
$$
= 5
$$
$$
r = \frac{-(-1) \pm\sqrt{\Delta}}{2(1)}
$$
$$
= \frac{1 \pm\sqrt{5}}{2}
$$

Isso quer dizer que (2) tem duas raízes diferentes:

$$
r_1 = \frac{1 + \sqrt{5}}{2}
$$
$$
r_2 = \frac{1 - \sqrt{5}}{2}
$$


Por $r_1\neq r_2$, a equação característica (2) implicará
na seguinte {\bf solução} para a recorrência (1):

$$
f(n) = a*(r_1)^n +b*(r_2)^n
\text{, sendo $a$ e $b$ constantes}
$$

Ou seja, a solução seria:

\begin{align}
f(n) =
a\left( \frac{1 + \sqrt{5}}{2} \right)^n
+
b\left( \frac{1 - \sqrt{5}}{2} \right)^n
\end{align}

Apesar de $a$ e $b$ serem constantes, tais valores ainda nos
são desconhecidos. E uma forma de encontrá-los seria criando
um sistema de equações composto por $f(0)$ e $f(1)$,
para garantir que a solução (3) satisfaça ambos os casos base
de (1) simultaneamente.
Logo, temos:

$$
\begin{cases}
a\left( \frac{1 + \sqrt{5}}{2} \right)^{\bf 0}
+
b\left( \frac{1 - \sqrt{5}}{2} \right)^{\bf 0}
=
f({\bf 0})
\\
a\left( \frac{1 + \sqrt{5}}{2} \right)^{\bf 1}
+
b\left( \frac{1 - \sqrt{5}}{2} \right)^{\bf 1}
=
f({\bf 1})
\end{cases}
$$

(e, por (1), já sabemos qual a imagem de $f$ quando $n=0$ e $n=1$)

$$
\begin{cases}
a(1)
+
b(1)
=
0
\\
a\left( \frac{1 + \sqrt{5}}{2} \right)
+
b\left( \frac{1 - \sqrt{5}}{2} \right)
=
1
\end{cases}
$$

\begin{align}
\begin{cases}
{\bf a + b = 0}
\\
{\bf a\left( \frac{1 + \sqrt{5}}{2} \right)
+
b\left( \frac{1 - \sqrt{5}}{2} \right)
=
1}
\end{cases}
\end{align}

Resolveremos esse sistema (4) usando a técnica de {\bf substituição}.
A começar pela primeira equação:

$$
a + b = 0
$$
$$
{\bf a = -b}
$$

Usaremos essa igualdade para retirar $a$ da segunda equação:

$$
{\bf -b}\left( \frac{1 + \sqrt{5}}{2} \right)
+
b\left( \frac{1 - \sqrt{5}}{2} \right)
=
1
$$

(e deixaremos o fator comum $b$ em evidência para simplificar)

$$
b\left[ - \left( \frac{1 + \sqrt{5}}{2} \right)
+
\left( \frac{1 - \sqrt{5}}{2} \right) \right]
=
1
$$

(finalizaremos executando a soma de frações e isolando $b$)


$$
b\left[ \frac{-(1 + \sqrt{5}) + (1 - \sqrt{5}) }{2} \right] = 1
$$
$$
b\left[ \frac{{\bf - 1} - \sqrt{5}\ {\bf + 1} - \sqrt{5} }{2} \right] = 1
$$
$$
b\left[ \frac{- \sqrt{5} - \sqrt{5} }{2} \right] = 1
$$
$$
b\left[ \frac{- {\bf 2}\sqrt{5} }{\bf 2} \right] = 1
$$
$$
b\left[ - \sqrt{5} \right] = 1
$$
$$
{\bf b = -\frac{1}{\sqrt{5}}}
$$

Agora $b$ teve seu valor descoberto, e será usado na
primeira equação de (4) para encontrar $a$:

$$
a = -b
$$
$$
a = -\left( - \frac{1}{\sqrt{5}} \right)
$$
$$
{\bf a = \frac{1}{\sqrt{5}}}
$$

Agora, já sabemos os valores das constantes $a$ e $b$ da função (3)
e das raízes $r_1$ e $r_2$ da característica (2). São dados o
suficiente para finalizar a solução (3).
Temos:

$$
f(n) =
\left( \frac{1}{\sqrt{5}} \right)
\left( \frac{1 + \sqrt{5}}{2} \right)^n
+
\left( -\frac{1}{\sqrt{5}} \right)
\left( \frac{1 - \sqrt{5}}{2} \right)^n
$$

(finalizamos pondo fator comum $\frac{1}{\sqrt{5}}$ em
evidência para simplificar)


$$
\bf
f(n) =
\frac{1}{\sqrt{5}}
\left[
\left( \frac{1 + \sqrt{5}}{2} \right)^n
-
\left( \frac{1 - \sqrt{5}}{2} \right)^n
\right]
$$

Esta é a {\bf solução} da relação (1) e pode ser provada
por indução.

\end{document}
